\chapter{Media}\label{chapter-media}

\begin{lstlisting}[language=python, title=run\_test.sh, frame=tb]
  "media_input" : {
      "#import" : "$MEDIA_DIR", 
      "#code_generate" : 1,   
      "#in_3lay_file" : "${IN_MEDIA_3LAY_FILE}", # *.md3lay
      "in_3grd_file" : "${IN_MEDIA_3GRD_FILE}",  # *.md3grd
      "equivalent_medium_method" : "har",       # default: "loc"
  },
  "is_export_media" : 1,                         # if export media                   
  "media_export_dir"  : "$MEDIA_DIR",                   
\end{lstlisting}

In the configuration of media, if \texttt{in\_3lay\_file} is selected, there need to be a .md3lay file, and the file format is shown in Section \ref{media-layer2model}. If \texttt{in\_3grd\_file} is selected, there need to be a .md3grd file, and the file format is shown in Section \ref{media-grid2model}.

%===================================================================
\section{Layer to Model} \label{media-layer2model}
The \textbf{media\_el\_iso\_layer2model} function is used to discretize the given layer model to the grid model. It provides two medium parameterization method: using the local point values (\texttt{loc}), and volume arithmetic and harmonic averaging method (\texttt{har}) \citep{moczo_3d_2002,moczo_finite-difference_2014}.

\subsection{File Format (.md3lay)}

\paragraph{3D Layer Velocity Model File (\texttt{.md3lay})\\}~

The following description ignores comment lines and blank lines.
\begin{itemize}
 \item The first line is the number of interface (\texttt{NI}).
 \item {The second line is the information of the given interface mesh:
  
  \texttt{NX} ~~ \texttt{NY} ~~ \texttt{MIN\_X} ~~ \texttt{MIN\_Y} ~~ \texttt{SPACING\_X} ~~ \texttt{SPACING\_Y}

  \texttt{NX} and \texttt{NY} are the number of points along $x$ and $y$ direction;

  \texttt{MIN\_X} and \texttt{MIN\_Y} are the minimal $x$ and $y$ coordinates;

  \texttt{SPACING\_X} and \texttt{SPACING\_Y} are spacing between points along $x$ and $y$.
 }
 \item {
  After then, the elevation, velocity and density are given as:
  \begin{lstlisting}[language = C]
  for (ni=0; ni<NI; ni++)
    for (iy=0; iy<NY; iy++) {
      for (ix=0; ix<NX; ix++) { 
        fscanf(layer_file, "%f %f %f %f %f %f %f", 
               &elevation[ni][iy][ix],
               &vp[ni][iy][ix], &vp_grad[ni][iy][ix],
               &vs[ni][iy][ix], &vs_grad[ni][iy][ix],
               &rho[ni][iy][ix], &rho_grad[ni][iy][ix]);
      }
    }
  }
  \end{lstlisting}
 }
 For each interface (from the free surface to bottom), a set of elevation values (elevation),
 $v_p$ (vp), the gradient of $v_p$ (vp\_grad), $v_s$ (vs), the gradient of $v_s$ (vs\_grad), $\rho$ (rho) and the gradient of $\rho$ (rho\_grad) on the regular 2D grid is required. 

 The velocities and density below interface(x, y, elevation) are calculated by
 \begin{equation*}
    v_p^{grid~point} = vp + (elevation-z^{grid~point}) * vp\_grad.
 \end{equation*} 
\end{itemize}

\subsection{Example}

A model with a horizontal interface can be given as:

\begin{lstlisting}[language=python, title=test.md3lay, frame=tb]
# NI
  2
# NX   NY   MIN_X  MIN_Y  SPACING_X  SPACING_Y 
  2    2    0.0    0.0    2000.0     2000.0
# elevation  vp  vp_grad  vs  vs_grad  rho  rho_grad
# interface #1 free surface
    0.0    2500.0  0.0  1500.0  0.0  1500.0  0.0   
    0.0    2500.0  0.0  1500.0  0.0  1500.0  0.0
    0.0    2500.0  0.0  1500.0  0.0  1500.0  0.0
    0.0    2500.0  0.0  1500.0  0.0  1500.0  0.0
# interface #2 
 -1000.0   4000.0  0.0  2400.0  0.0  2400.0  0.0   
 -1000.0   4000.0  0.0  2400.0  0.0  2400.0  0.0
 -1000.0   4000.0  0.0  2400.0  0.0  2400.0  0.0
 -1000.0   4000.0  0.0  2400.0  0.0  2400.0  0.0 
\end{lstlisting}
We provide a more complex model in the \texttt{test/} directory.

%=============================================================
\section{Grid to Model} \label{media-grid2model}
The \textbf{media\_el\_iso\_grid2model} function is used to discretize the given grid model to the grid model. It also provides two medium parameterization method: using the local point values (\texttt{loc}), and volume arithmetic and harmonic averaging method (\texttt{har} in ) \citep{moczo_3d_2002,moczo_finite-difference_2014}.

\subsection{File Format (.md3grd)}
The following description ignores comment lines and blank lines.
\begin{itemize}
 \item The first line is the number of layer (\texttt{NL}), if \texttt{NL} > 1, there is a designated interface.
 \item the next \texttt{NL} lines are the number of grids in the z-direction of each layer 
 \item {The third line is the information of the given interface mesh:
  
  \texttt{NX} ~~ \texttt{NY} ~~ \texttt{MIN\_X} ~~ \texttt{MIN\_Y} ~~ \texttt{SPACING\_X} ~~ \texttt{SPACING\_Y}

  \texttt{NX} and \texttt{NY} are the number of points along $x$ and $y$ direction;

  \texttt{MIN\_X} and \texttt{MIN\_Y} are the minimal $x$ and $y$ coordinates;

  \texttt{SPACING\_X} and \texttt{SPACING\_Y} are spacing between points along $x$ and $y$.
 }
 \item {
  After then, the elevation, velocity and density are given in every grid points:
  \begin{lstlisting}[language = C]
  for (ig=0; ig<ng_z; ig++)
    for (iy=0; iy<NY; iy++) {
      for (ix=0; ix<NX; ix++) { 
        fscanf(grid_file, "%f %f %f %f", &elevation[ig][iy][ix], 
              &vp[ig][iy][ix], &vs[ig][iy][ix], &rho[ig][iy][ix]);
      }
    }
  }
  \end{lstlisting}
 } 
\end{itemize}
The velocities and density are calculated by interpolation of the values at the given grid points.

\subsection{Example}
A model with a horizontal interface can be given as:
\begin{lstlisting}[language=python, title=test.md3lay, frame=tb]
# NL
  2
# How many z-grids are in each layer
  2
  2

# NX   NY   MIN_X  MIN_Y  SPACING_X  SPACING_Y 
  2    2    0.0    0.0    2000.0     2000.0
# elevation   vp     vs     rho  
# z-grid #1: Top - free surface
    0.0    2500.0  1500.0  1500.0 
    0.0    2500.0  1500.0  1500.0 
    0.0    2500.0  1500.0  1500.0 
    0.0    2500.0  1500.0  1500.0 
# z-grid #2 
 -1000.0   2500.0  1500.0  2400.0   
 -1000.0   2500.0  1500.0  2400.0
 -1000.0   2500.0  1500.0  2400.0
 -1000.0   2500.0  1500.0  2400.0 
# z-grid #3 (the elevation needs to be the same as #2)
 -1000.0   4000.0  2400.0  2400.0   
 -1000.0   4000.0  2400.0  2400.0
 -1000.0   4000.0  2400.0  2400.0
 -1000.0   4000.0  2400.0  2400.0 
# z-grid #4 
 -2000.0   4000.0  2400.0  2400.0   
 -2000.0   4000.0  2400.0  2400.0
 -2000.0   4000.0  2400.0  2400.0
 -2000.0   4000.0  2400.0  2400.0  
\end{lstlisting}
if \texttt{NL} > 1, there is a designated interface; and the elevation of the ng[il]+1 needs to be the same as ng[il]. The equivalent medium parameterization method can be applied on this interface.
We provide a more complex model in the \texttt{test/} directory.
