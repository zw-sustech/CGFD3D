\chapter{The Structure of Code and Compilation}\label{chap_compile}
%=============================================================
\section{The Structure of Code} 
%=============================================================
CGFD3D-elastic package is download as a directory with the follow subdirectories.
\renewcommand*\DTstylecomment{\rmfamily\color{blue}}
\renewcommand*\DTstyle{\ttfamily\color{red}}
\dirtree{%
.1 CGFD3D-elastic.
.2 doc/\DTcomment{This USER\_MANUAL can be found here}.
.2 forward/\DTcomment{Source code for programs}.
.2 Makefile.
.2 mfiles/\DTcomment{Drawing code in MATLAB}.
.2 run\_make.mars\DTcomment{The bash-script to run to make on mars server}.
.2 test/\DTcomment{Code for test}.
.2 example/\DTcomment{Code for preparing and running script}.
.2 lib/\DTcomment{Library used by the programs}.
.2 media/\DTcomment{Source code related to media}.
.2 pyfiles/\DTcomment{Drawing code in Python}.
.2 run\_make.server1\DTcomment{The bash-script to run to make on server1 server}.
}
%=============================================================
\section{Compilation and Linking} \label{compile}
%=============================================================
CGFD3D-elastic package is written in C and C++, so makesure C and C++ compilers are installed on your system. And you need to install MPI and NETCDF. MATLAB is also needed for generating data and displaying the results. \\
Make clean is suggested before making by typing \\
~\\
\verb|> make distclean| \\
~\\
Then set the ROOT variable. Here we provide the bash script \verb|run_make.server1| (Listing \ref{make}) for making code. You need to provide \verb|MPI_ROOT| and \verb|NETCDFROOT| in \verb|run_make.server1| to adapt to the system you are using, and add mpi \verb|bin| file to PATH. The  target executable is \verb|main_curv_col_el_3d|. 
\begin{lstlisting}[language=json,
	caption=run\_make.server1,
	label={make},
	frame=tb]
#-- add mpi to PATH if module is not used, on server1
MPI_ROOT=/share/apps/gnu-4.8.5/mpich-3.3/
export PATH=$MPI_ROOT/bin:$PATH
export NETCDFROOT=/share/apps/gnu-4.8.5/disable-netcdf-4.4.1
	
echo "mpicc and mpicxx will invoke:"
mpicc -show
mpicxx -show
	
echo
echo "start to make ..."
make -f Makefile
\end{lstlisting}
After setting ROOT variable, run the script with the command \\
~\\
\verb|> ./run_make.server1| \\
~\\
If module is used on your system, we provide another bash script \verb|run_make.mars|. Before running the script, you need to load compiler, mpi, netcdf and matlab first. After that, run the script\\
~\\
\verb|> ./run_make.mars| \\
~\\
The object files and executables will be generated.\\
~\\
Some useful make commands: \\
\verb|make cleanall | : remove all object files and executables. \\
\verb|make cleanexe | : remove all executables. \\
\verb|make cleanobj | : remove all objects files. 


